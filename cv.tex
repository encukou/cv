
%
% Copyright (c) 2010-2012 Petr Viktorin
% The layout/design of this CV is licensed under a  licensed under the Creative
% Commons Attribution-ShareAlike 3.0 Unported License, available at
% http://creativecommons.org/licenses/by-sa/3.0/
%
% Use Lualatex to typeset the document
%
% The content is, of course, quite entirely mine.
%
% Requested attribution:
% please link to http://github.com/encukou/cv (or your fork).
%
% For sending your finished CV to companies/headhunters, no attribution
% is required on their copy.
%

\documentclass[a4paper,10pt]{article}
\usepackage[utf8]{luainputenc}
\usepackage[colorlinks=true, allcolors=blue]{hyperref}

\providecommand\tr[2]{#2}
\tr{}{\usepackage[czech]{babel}}
\providecommand\personalinfoAddress{{\huge\tt <do not typeset cv.tex directly>}}
\providecommand\personalinfoPhone{\vspace{20em}}
\providecommand\includeinfo{\newcommand\personalinfoAddress{}
\newcommand\personalinfoPhone{\textit{(available by request)}}
}
\providecommand\includefullinfo{
    \IfFileExists{personalinfo.tex}{\input{personalinfo.tex}}{\newcommand\personalinfoAddress{}
\newcommand\personalinfoPhone{\textit{(available by request)}}
}}
\includeinfo

%\usepackage[T1, IL2]{fontenc}
\usepackage[nodayofweek]{datetime}

\usepackage{fouriernc}
\usepackage{tgpagella}
\usepackage{color}
\usepackage[activate=true]{microtype}
\usepackage{amssymb}
\usepackage{fontspec}
\usepackage{siunitx}
\usepackage{enumitem}

\usepackage{calc}
\newdimen\runinlength
\setlength\runinlength{3cm+1em}
\newdimen\centeroffset
\setlength\centeroffset{-\runinlength/4}
\usepackage[a4paper,margin=2cm,left=2cm+\runinlength, right=3.5cm]{geometry}

%\setmainfont[Numbers=OldStyle]{TeX Gyre Pagella}

\pagestyle{empty}



\parindent=0cm
\newcommand\runin[1]{\par\makebox[0cm]{\hspace{-\runinlength}\makebox[3cm]{\hspace{\fill}#1}\hspace{\fill}}}

\newcommand\miniheader[1]{{\bf#1}\par}
\newcommand\miniheaderii[2]{{\bf#1} #2\par}

\newcommand\header[2]{\vspace{1em}\runin{#1}\miniheader{#2}}
\newcommand\headerii[3]{\vspace{1em}\runin{#1}\miniheaderii{#2}{#3}}

\newcommand\positionheader[2]{\hspace{0em}{\bf#1} {#2}\par}

\newcommand\centered[1]{
  \hspace{\centeroffset}\begin{minipage}{\textwidth}
  \begin{center}
  #1
  \end{center}
  \end{minipage}
}

\renewenvironment{itemize}{
  \setlist{nosep}
  \newdimen\itemizeskip
  \setlength\itemizeskip{.5ex}
  \renewcommand\item[1][$\bullet$]{\vspace{\itemizeskip}\runin{##1}}
  \newdimen\itemizetopskip
}{
  \vspace{\itemizeskip}
}

\renewcommand\section[1]{\vspace{2.5em}{\hspace{-2em}\Large#1}}

\newcommand\pep[1]{\href{https://peps.python.org/#1}{#1}}
\newcommand\pepd[2]{\href{https://peps.python.org/#1}{PEP #1} -- #2}

\begin{document}

\centered{
{\Large\sc Petr Viktorin}


\small
\vspace{1em}

\personalinfoAddress

\href{mailto:encukou@gmail.com}{encukou@gmail.com} \\
Brno, Czech Republic \\

}

\section{\tr{Work Experience}{Praxe}}

\header{\tr{2012 to date}{2012---\phantom{000}}}{Red Hat}
\positionheader{Principal Software Engineer}{(\tr{since 2022}{2022---})}
\vspace{1em}
%\positionheader{Senior Software Engineer}{(\tr{2016 to 2022}{2016---2022})}
\tr{
    As part of the \emph{Python maintenance team}, I help integrate Python and
    Python-based software into RPM-based Linux distributions.
    I've co-authored the 2021 rewrite of the \href{https://docs.fedoraproject.org/en-US/packaging-guidelines/Python/}{Fedora Python Packaging Guidelines}
    and the associated \href{https://src.fedoraproject.org/rpms/pyproject-rpm-macros}{\texttt{pyproject} RPM macros},
    aiming to reuse upstream packaging metadata and align Fedora with the \href{https://pypi.org/}{PyPI} ecosystem.

    I've coordinated porting thousands of Fedora packages from Python~2
    to Python~3.

    I was also allowed to spend significant work time contributing to CPython.
}{
    Správa jazyka Python
}

\vspace{1em}

%\positionheader{Software Engineer}{(\tr{2014 to 2016}{2014---2016})}
%\positionheader{Associate Software Engineer}{(\tr{2012 to 2014}{2012---2014})}
\tr{
    Before 2015, I worked on the \href{http://freeipa.org/}{FreeIPA} project, a part of Red Hat Identity Management.
    My responsibilities included feature development, refactoring, and bug fixing.
    I~also lead the upstream integration testing effort.\\
    Technologies used: Python, LDAP, Nose, Jenkins CI
}{
    Práce na projektu \href{http://freeipa.org/}{FreeIPA}\\
    s použitím technologií Python, LDAP, Kerberos
}

\header{2006--2009}{\tr{Programmer at DM Juventus}{Programátor v DM Juventus}}
\vspace{1em}
\tr{
    I have developed a complete information system for a small hotel,\\
    using C++/Qt and MySQL.
}{
    Návrh a vývoj informačního systému pro rezervace a správu hotelových hostů
    s~použitím technologií C++/Qt a MySQL.
}

\section{\tr{CPython Experience}{Praxe}}

\header{2018 to date}{CPython core developer}
    I work on the Python C API, improving the stable ABI, improving support
    for multiple interpreters and making the API more friendly for languages
    other than~C. I have (co-)authored several Python Enhancement Proposals:
    \begin{itemize}
    \item[]\pepd{489}{Multi-phase extension module initialization}
    \item[]\pepd{573}{Module State Access from C Extension Methods}
    \item[]\pepd{630}{Isolating Extension Modules}
    \item[]\pepd{652}{Maintaining the Stable ABI}
    \item[]\pepd{687}{Isolating modules in the standard library}
    \item[]\pepd{689}{Unstable C API tier}
    \item[]\pepd{697}{Limited C API for Extending Opaque Types}
    \end{itemize}

    I made several contributions related to my job at Red Hat,
    around packaging and security issues:

    \begin{itemize}
    \item[]\pepd{394}{The “python” Command on Unix-Like Systems}
    \item[]\pepd{627}{Recording installed projects}
    \item[]\pepd{672}{Unicode-related Security Considerations for Python}
    \item[]\pepd{706}{Filter for {\texttt{tarfile.extractall}}}
    \end{itemize}

\header{2022}{Python Steering Council member}
    I was elected to the highest decision-making body in
    the Python project.

\section{\tr{Education}{Vzdělání}}
\vspace{1em}
\runin{2012}\miniheaderii{MSc,}{University of Eastern Finland, School of Computing}
\runin{2009}\miniheaderii{Bc.,}{\tr{Brno University of Technology, Faculty of Informatics}{Fakulta informatiky, Vysoké učení technické v Brně}}

\section{\tr{Related Activities}{Aktivity a zájmy}}
\vspace{1em}
\begin{itemize}
 \item[Open source]%
    \tr{%
        I contribute to various open-source projects at \href{http://github.com/encukou}{\texttt{github.com/encukou}}
    }{%
        Přispívám do několika open-source projektů na \href{http://github.com/encukou}{\texttt{github.com/encukou}}
    }
 \item[Conference talks]%
    I've spoken at several conferences. Here are the highlights.
    \begin{itemize}
    \item[]2021 -- \href{https://www.youtube.com/watch?v=pJlqB-PA96E}{Python in Fedora} (PackagingCon)
    \item[]2019 -- \href{https://cz.pycon.org/2019/programme/talks/18/}{Building an async event loop} (PyCon CZ)
    \item[]2018 -- \href{https://ep2018.europython.eu/conference/talks/bytecodes-and-stacks-a-look-at-cpythons-compiler-and-its-execution-model.html}{Bytecodes and stacks} (EuroPython)
    \item[]2017 -- \href{https://cz.pycon.org/2017/speakers/detail/talk/14/}{How we started teaching Python} (PyCon CZ)
    \item[]2016 -- \href{https://cz.pycon.org/2016/speakers/workshops/#petr-viktorin}{Turn On the Lights with MicroPython} (PyCon CZ workshop)
    \item[]2015 -- \href{https://ep2015.europython.eu/conference/talks/import-deep-dive.html}{Import Deep Dive} (EuroPython)
    \item[]2014 -- \href{https://ep2014.europython.eu/en/schedule/sessions/123/}{The Magic of Attribute Access} (EuroPython)
    \item[]2013 -- \href{https://ep2013.europython.eu/conference/talks/terminals-command-lines-and-text-interfaces.html}{Terminals, command lines, and text interfaces } (EuroPython)
    \end{itemize}
 \item[Pyvec]%
    \tr{%
        As a member of
    }{%
        Jsem členem sdružení
    }
        \href{http://pyvec.org}{Pyvec.org}%
    \tr{%
        , a non-profit that promotes Python in the Czech Republic,
        I helped organize \href{https://pyvo.cz/en/}{Pyvo} meetups in Brno,
        designed and taught free beginner courses for PyLadies CZ (and \href{https://naucse.python.cz/2021/online-jaro/}{anyone online}),
        and helped at PyCon~CZ.
    }{%
        , které podporuje jazyk Python v~České republice.
        Jsem spoluorganizátor brněnských pythonových srazů.
    }
\end{itemize}

\end{document}

