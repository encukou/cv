
%
% Copyright (c) 2010-2012 Petr Viktorin
% The layout/design of this CV is licensed under a  licensed under the Creative
% Commons Attribution-ShareAlike 3.0 Unported License, available at
% http://creativecommons.org/licenses/by-sa/3.0/
%
% Use Lualatex to typeset the document
%
% The content is, of course, quite entirely mine.
%
% Requested attribution:
% please link to http://github.com/encukou/cv (or your fork).
%
% For sending your finished CV to companies/headhunters, no attribution
% is required on their copy.
%

\documentclass[a4paper,10pt]{article}
\usepackage[utf8]{luainputenc}
\usepackage[pdfborder={0 0 0}]{hyperref}

\providecommand\tr[2]{#2}
\tr{}{\usepackage[czech]{babel}}
\providecommand\personalinfoAddress{{\huge\tt <do not typeset cv.tex directly>}}
\providecommand\personalinfoPhone{\vspace{20em}}
\providecommand\includeinfo{\newcommand\personalinfoAddress{}
\newcommand\personalinfoPhone{\textit{(available by request)}}
}
\providecommand\includefullinfo{
    \IfFileExists{personalinfo.tex}{\input{personalinfo.tex}}{\newcommand\personalinfoAddress{}
\newcommand\personalinfoPhone{\textit{(available by request)}}
}}
\includeinfo

%\usepackage[T1, IL2]{fontenc}
\usepackage[nodayofweek]{datetime}

\usepackage{fouriernc}
\usepackage{tgpagella}
\usepackage{color}
\usepackage[activate=true]{microtype}
\usepackage{amssymb}
\usepackage{fontspec}

\usepackage{calc}
\newdimen\runinlength
\setlength\runinlength{3cm+1em}
\newdimen\centeroffset
\setlength\centeroffset{-\runinlength/4}
\usepackage[a4paper,margin=2cm,left=2cm+\runinlength, right=3.5cm]{geometry}

%\setmainfont[Numbers=OldStyle]{TeX Gyre Pagella}

\pagestyle{empty}

\parindent=0cm
\newcommand\runin[1]{\par\makebox[0cm]{\hspace{-\runinlength}\makebox[3cm]{\hspace{\fill}#1}\hspace{\fill}}}

\newcommand\miniheader[1]{{\bf#1}\par}

\newcommand\header[2]{\vspace{1em}\runin{#1}\miniheader{#2}}

\newcommand\positionheader[2]{\hspace{1em}{\bf#1} {#2}\par}

\newcommand\centered[1]{
  \hspace{\centeroffset}\begin{minipage}{\textwidth}
  \begin{center}
  #1
  \end{center}
  \end{minipage}
}

\renewenvironment{itemize}{
  \newdimen\itemizeskip
  \setlength\itemizeskip{1ex}
  \renewcommand\item[1][$\bullet$]{\vspace{\itemizeskip}\runin{##1}}
  \newdimen\itemizetopskip
  \setlength\itemizetopskip{1em-\itemizeskip}
  \vspace{\itemizetopskip}
}{
}

\renewcommand\section[1]{\vspace{2.5em}{\hspace{-2em}\Large#1}}

\begin{document}

\centered{
{\Large\sc Petr Viktorin}


\small
\vspace{1em}

\personalinfoAddress

\vspace{.5em}
\begin{tabular}{rl}
\tr{Birth Date}{Datum narození} &
    \tr{14$^{\textrm\small{th}}$ of January, 1987}{14. ledna 1987} \\
E-mail &
    \href{mailto:encukou@gmail.com}{encukou@gmail.com} \\
\tr{Phone}{Telefon} &
    \personalinfoPhone
\end{tabular}

}

\section{\tr{Work Experience}{Praxe}}

\header{\tr{2012 to date}{2012---\phantom{000}}}{Red Hat}
\positionheader{Principal Software Engineer}{(\tr{2022 to date}{2022---})}
\positionheader{Senior Software Engineer}{(\tr{2016 to 2022}{2016---2022})}
\tr{
    Python maintenance
}{
    Správa jazyka Python
}

\positionheader{Software Engineer}{(\tr{2014 to 2016}{2014---2016})}
\positionheader{Associate Software Engineer}{(\tr{2012 to 2014}{2012---2014})}
\tr{
    I worked on the \href{http://freeipa.org/}{FreeIPA} project, a part of Red Hat Identity Management.\\
    My responsibilities include feature development, refactoring, and bug fixing.\\
    I~also lead the upstream integration testing effort.\\
    Technologies used: Python, LDAP, Nose, Jenkins CI
}{
    Práce na projektu \href{http://freeipa.org/}{FreeIPA}\\
    s použitím technologií Python, LDAP, Kerberos
}

\header{2006--2009}{\tr{Programmer at DM Juventus}{Programátor v DM Juventus}}
\tr{
    I have developed a complete information system for a small hotel.\\
    Technologies used: C++/Qt, MySQL.
}{
    Návrh a vývoj informačního systému pro rezervace a správu hotelových hostů
    s~použitím technologií C++/Qt a MySQL.
}

\section{\tr{Education and Qualifications}{Vzdělání}}

\header{2012}{Red Hat Certified Engineer}
\tr{Certification number}{Certifikát číslo}
    \href{https://www.redhat.com/wapps/training/certification/verify.html?certNumber=120-204-953}{%
        \texttt{120-204-953}}

\header{\tr{2009--2012}{2009--2012}}{MSc, IMPIT program, University of Eastern Finland}
\tr{
    I have led a 2-credit \emph{Python Workshop} in 2010
}{
    V roce 2010 jsem vedl kurz \emph{Python Workshop}
}

\header{2006--2009}{Bc., \tr{Faculty of Informatics, Brno University of Technology}{Fakulta informatiky, Vysoké učení technické v Brně}}

\header{2004--2006}{Farragut High School, Tennessee, USA}
\tr{
    Graduated from an English-language high school.
}{
    Úspěšné studium v angličtině.
}

\section{\tr{Activities and Interests}{Aktivity a zájmy}}
\begin{itemize}
 \item[\tr{Programming}{Programování}]%
    \tr{%
        I program in Python.
        I also know my way around C, C++/Qt, and Java.
    }{%
        Programuji v~jazyce Python.
        Rozumím ale také jazykům C, C++/Qt, a Java.
    }
 \item[Open source]%
    \tr{%
        I contribute to several open-source projects at \href{http://github.com/encukou}{\texttt{github.com/encukou}}
    }{%
        Přispívám do několika open-source projektů na \href{http://github.com/encukou}{\texttt{github.com/encukou}}
    }
 \item[Pyvec]%
    \tr{%
        I am a member of the
    }{%
        Jsem členem sdružení
    }
        \href{http://pyvec.org}{\texttt{Pyvec.org}}
    \tr{%
        organization, which promotes the Python language in the Czech Republic.
        I help organize the “Pyvo” meetups in Brno.
    }{%
        , které podporuje jazyk Python v~České republice.
        Jsem spoluorganizátor brněnských pythonových srazů.
    }
 \item[Stack Overflow]%
    \tr{%
        I have 10,000+ reputation on the question-and-answer site
    }{%
        Na stránce
    }%
        \href{http://stackoverflow.com/users/99057/petr-viktorin}{\texttt{stackoverflow.com}}%
    \tr{%
        .%
    }{
        mám reputaci přes 10\,000.
    }%
 \item[\tr{Other interests}{Další zájmy}]%
    \tr{%
        I enjoy hobby electronics, traveling, the outdoors, and puzzle hunt games.
    }{%
        Zajímám se o~elektroniku, cestuji, hraji šifrovací hry.
    }
\end{itemize}

\section{\tr{Languages Spoken}{Jazyky}}
\begin{itemize}
  \item \tr{Fluent English and Czech}{Plynná angličtina a čeština}
  \item \tr{Beginner German}{Začátečnická němčina}
\end{itemize}

\vfill
{
\small\it\hfill
\tr{
    Current as of \longdate\today
}{
    \newdateformat{csdate}{\THEDAY. {\monthname} \THEYEAR}
    \csdate\today
}
}

\end{document}

